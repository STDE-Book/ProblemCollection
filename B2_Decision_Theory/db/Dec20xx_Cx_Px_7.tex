\ifspanish

\question Se tiene un problema de clasificación binaria bidimensional definido por las siguientes verosimilitudes:
 $$ \begin{array}{l} 
					   p_{X_1,X_2|H}(x_1,x_2 | 0) = G \left( {\bf 0}, \left[ \begin{array}{cc}  
					   1 & \rho \\ \rho & 1					   
					    \end{array}  \right] \right)\\ \\
					  p_{X_1,X_2|H}(x_1,x_2 | 1) = G \left( {\bf m}, \left[ \begin{array}{cc}  
					   1 & \rho \\ \rho & 1					   
					    \end{array}  \right] \right)	  
  \end{array}$$
Represéntese en el plano $X_1-X_2$ la frontera de decisión que proporciona el decisor MAP cuando se satisfacen las siguientes condiciones: $P_H(0)= P_H (1)$, $v_0=v_1$ y $\rho=0$.
Indique cómo se modificaría la frontera anterior si:
\begin{parts}
\part Las probabilidades a priori fuesen $P_H(0)= 2 P_H(1)$.
\part Se incrementase el valor de $\rho$.
\end{parts}
\begin{solution} 
La frontera es la mediatriz de la recta que une los centros de las dos gaussianas.
\begin{parts}
\part Si la $P_H(0)$ es mayor, la recta se desplaza hacia la verosimilitud de $H=1$, es decir, hacia el punto ${\bf m}$.
\part No varía.
\end{parts}
\end{solution}

\else

\question Let the following likelihoods characterize a bidimensional binary decision problem:
 $$ \begin{array}{l} 
					   p_{X_1,X_2|H}(x_1,x_2 | 0) = G \left( {\bf 0}, \left[ \begin{array}{cc}  
					   1 & \rho \\ \rho & 1					   
					    \end{array}  \right] \right)\\ \\
					  p_{X_1,X_2|H}(x_1,x_2 | 1) = G \left( {\bf m}, \left[ \begin{array}{cc}  
					   1 & \rho \\ \rho & 1					   
					    \end{array}  \right] \right)	  
  \end{array}$$
Plot in plane $X_1-X_2$ the decision border given by the MAP classifier, if the following conditions hold: $P_H(0)= P_H (1)$, $v_0=v_1$ and $\rho=0$. Indicate how that decision border would change if:
\begin{parts}
\part The {\em a priori} probabilities were $P_H(0)= 2 P_H(1)$.
\part The value of $\rho$ were increased.
\end{parts}

\begin{solution}
The decision border is the bisector of the segment joining the means of both Gaussian distributions.
\begin{parts}
\part If $P_H(0)$ gets larger, then the decision border is shifted towards the likelihood of hypothesis $H=1$, i.e., towards ${\bf m}$.
\part The decision border does not change.
\end{parts}
\end{solution}

\fi