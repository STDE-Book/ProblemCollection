\ifspanish

\question En un problema de clasificación binaria se sabe que las observaciones presentan las siguientes distribuciones:
 $$					  \begin{array}{ll}
					  p_{X|H}(x|0) = \exp (-x), & \quad	x>0 \\
					  p_{X|H}(x|1) =  a \exp (-a x), & \quad	x>0\\				
					  \end{array}	  $$
con $a >1$. Para la toma de la decisión se dispone de un conjunto de $K$ observaciones independientes tomadas bajo la misma hipótesis:  $\left\lbrace X^{(k)} \right\rbrace_{k=1}^K$.
\begin{parts}
\part Obténgase el decisor ML basado en el conjunto de observaciones  $\left\lbrace X^{(k)} \right\rbrace_{k=1}^K$ y compruébese, a partir de resultado obtenido, que $T=\sum_{k=1}^K X^{(k)} $ es un estadístico suficiente para la decisión.
Considérese para el resto del ejercicio $K=2$. 
\part Calcúlense las verosimilitudes del estadístico $T$, $p_{T|H}(t|0)$ y $p_{T|H}(t|1)$.
\part Calcúlense, en función del valor de $\eta$, las $P_{\rm FA}$ y $P_{\rm M}$ del decisor de umbral
$$ t \dceroduno \eta$$
\part Represéntese de forma aproximada la curva ROC del decisor anterior, indicando:
\begin{itemize}
\item Cómo se desplaza el punto de trabajo al aumentar $\eta$.
\item Cómo se modificaría la curva ROC si creciese el número de observaciones disponibles ($K$).
\item Cómo se modificaría la curva ROC al incrementar el valor de $a$.
\end{itemize}

\end{parts}

\begin{solution}
\begin{parts}
\part $t \dceroduno \displaystyle \frac{K \ln a}{a-1}$ \\
\part $  \begin{array}{ll}
					  p_{T|H}(t|0) = t\exp (-t), & \quad	t>0 \\
					  p_{T|H}(t|1) =  a^2 t \exp (-at), & \quad	t>0\\				
					  \end{array}	  $
\part $P_{\rm FA} = 1- (\eta +1) \exp(-\eta)  \quad P_{\rm M} = (a\eta +1) \exp(-a\eta)$ 
\part 
\begin{itemize}
\item Para $\eta = 0, \; P_{\rm FA} = P_{\rm D} = 0$; Para $\eta \rightarrow \infty, \; P_{\rm FA} = P_{\rm D} = 1$.
\item Si crece el número de observaciones, necesariamente debe mejorar la curva ROC.
\item Si crece el valor de $a$, también debe mejorar la curva ROC. Una comprobación rigurosa sería:
$\displaystyle \frac{\partial P_{\rm M}}{\partial a}=-a \eta^2 exp(- a \eta ) < 0$, luego la probabilidad de pérdida decrece al aumentar el valor de $a$.
\end{itemize}
\end{parts}
\end{solution}

\else

\question Consider a binary classification problem where observations are distributed according to:
 $$					  \begin{array}{ll}
					  p_{X|H}(x|0) = \exp (-x), & \quad	x>0 \\
					  p_{X|H}(x|1) =  a \exp (-a x), & \quad	x>0\\				
					  \end{array}	  $$
with $a >1$. For the decision, $K$ independent observations, taken under the same hypothesis, are available: $\left\lbrace X^{(k)} \right\rbrace_{k=1}^K$.
\begin{parts}
\part Obtain the ML classifier based on the set of observations $\left\lbrace X^{(k)} \right\rbrace_{k=1}^K$ and check, using such a classifier, that $T=\sum_{k=1}^K X^{(k)}$ is a sufficient statistic for the decision.
\end{parts}
Consider $K=2$ for the rest of the exercise. 
\begin{parts}
\setcounter{partno}{1}
\part Find the likelihoods in terms of the sufficient statistic $T$, $p_{T|H}(t|0)$ and $p_{T|H}(t|1)$.
\part Calculate $P_{\rm FA}$ and $P_{\rm M}$ for the following threshold decision maker, as a function of $\eta$:
$$ t \dceroduno \eta$$
\part Provide an approximate plot of the ROC curve for the previous decision maker, indicating:
\begin{itemize}
\item How the operation point moves when increasing $\eta$.
\item How the ROC curve would change if we had access to a larger number of observations $K$.
\item How the ROC curve changes as the value of $a$ increases.
\end{itemize}

\end{parts}

\begin{solution}
\begin{parts}
\part $t \dceroduno \displaystyle \frac{K \ln a}{a-1}$ \\
\part $  \begin{array}{ll}
					  p_{T|H}(t|0) = t\exp (-t), & \quad	t>0 \\
					  p_{T|H}(t|1) =  a^2 t \exp (-at), & \quad	t>0\\				
					  \end{array}	  $
\part $P_{\rm FA} = 1- (\eta +1) \exp(-\eta)  \quad P_{\rm M} = (a\eta +1) \exp(-a\eta)$ 
\part 
\begin{itemize}
\item For $\eta = 0, \; P_{\rm FA} = P_{\rm D} = 0$; for $\eta \rightarrow \infty, \; P_{\rm FA} = P_{\rm D} = 1$.
\item If the number of observations increases, then necessarily the performance of the classifier should improve (the area below the ROC curve increases).
\item The same occurs if the value of $a$ is increased.  A rigorous demonstration would be:
$\displaystyle \frac{\partial P_{\rm M}}{\partial a}=-a \eta^2 exp(- a \eta ) < 0$, thus $P_M$ decreases as the value of $a$ is increased.
\end{itemize}
\end{parts}

\end{solution}

\fi