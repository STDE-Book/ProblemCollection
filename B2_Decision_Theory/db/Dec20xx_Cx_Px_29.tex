\ifspanish

\question Considere el problema de decisión binaria dado por las verosimilitudes
\begin{align*}
p_{\mathbf X|H}\left( \begin{bmatrix} x_1 \\ x_2 \end{bmatrix} \mid 0\right) 
	&\sim G\left(\begin{bmatrix} 0 \\ 0 \end{bmatrix}, 
	             \begin{bmatrix} 1 & 0 \\ 0 & 1 \end{bmatrix} \right), \\
p_{\mathbf X|H}\left( \begin{bmatrix} x_1 \\ x_2 \end{bmatrix} \mid 1\right) 
    &\sim G\left(\begin{bmatrix} 1 \\ 1 \end{bmatrix}, 
	             \begin{bmatrix} 1 & 0 \\ 0 & 1 \end{bmatrix} \right) 
\end{align*}

\begin{parts}
\part Obtenga la expresión del decisor ML y compruebe que para la toma de la decisión es suficiente conocer la variable $T = X_1 + X_2$.
\part Obtenga las densidades de probabilidad $p_{T|H}(t|0)$ y $p_{T|H}(t|1)$.
\part Calcule las probabilidades de falsa alarma y de pérdida a partir de las verosimilitudes obtenidas en el apartado anterior. Exprese el resultado haciendo uso de la función 
$$F(z)=\int_{-\infty}^z \! \frac{1}{\sqrt{2\pi}}\exp{\left(-\frac{u^2}{2}\right)} \, du $$
\end{parts}

\else

\question Consider a binary decision problem with likelihoods
\begin{align*}
p_{\mathbf X|H}\left( \begin{bmatrix} x_1 \\ x_2 \end{bmatrix} \mid 0\right) 
	&\sim G\left(\begin{bmatrix} 0 \\ 0 \end{bmatrix}, 
	             \begin{bmatrix} 1 & 0 \\ 0 & 1 \end{bmatrix} \right), \\
p_{\mathbf X|H}\left( \begin{bmatrix} x_1 \\ x_2 \end{bmatrix} \mid 1\right) 
    &\sim G\left(\begin{bmatrix} 1 \\ 1 \end{bmatrix}, 
	             \begin{bmatrix} 1 & 0 \\ 0 & 1 \end{bmatrix} \right) 
\end{align*}

\begin{parts}
\part Obtain the ML classifier, and check that the knowledge of $T = X_1 + X_2$ is sufficient for taking decisions.
\part Obtain the conditional probability density functions $p_{T|H}(t|0)$ and $p_{T|H}(t|1)$.
\part Calculate the false alarm and missing probabilities using the likelihoods of the previous section. Express your result by means of function
$$F(z)=\int_{-\infty}^z \! \frac{1}{\sqrt{2\pi}}\exp{\left(-\frac{u^2}{2}\right)} \, du $$
\end{parts}

\fi

%%%%%%%%%%%%%%%%
\begin{solution}
\begin{parts}
\part \ifspanish El clasificador ML está dado por \else The ML classifier is given by \fi
\begin{align*}
p_{{\bf X}|H}({\bf x}\mid 1) & \dunodcero p_{{\bf X}|H}({\bf x}\mid 1)  \\
	\Leftrightarrow \quad & 
		\frac1{2\pi}
	    \exp\left(-\frac12 \left({\bf x}-\begin{bmatrix} 1 \\ 1 \end{bmatrix}\right)^\top  
	                       \left({\bf x}-\begin{bmatrix} 1 \\ 1 \end{bmatrix}\right)\right)
	    \dunodcero
	    \frac1{2\pi}\exp\left(-\frac12 {\bf x}^\top{\bf x}\right) \\ 
	\Leftrightarrow \quad & 
	    - \left({\bf x}-\begin{bmatrix} 1 \\ 1 \end{bmatrix}\right)^\top  
	      \left({\bf x}-\begin{bmatrix} 1 \\ 1 \end{bmatrix}\right) 
		\dunodcero
		- {\bf x}^\top{\bf x}    \\
	\Leftrightarrow \quad & 
	    2 \begin{bmatrix} 1 \\ 1 \end{bmatrix}^\top {\bf x}
	    \dunodcero 
	 	\begin{bmatrix} 1 \\ 1 \end{bmatrix}^\top \begin{bmatrix} 1 \\ 1 \end{bmatrix}    \\
	\Leftrightarrow \quad &  x_1 + x_2 \dunodcero 1 
\end{align*}
\ifspanish Definiendo $t=x_1+x_2$, se obtiene el test equivalente 
\else Defining $t = x_1 + x_2$ we get the equivalent test \fi
\begin{align*}
t \dunodcero 1
\end{align*}

\part 
\ifspanish Dado que $T$ es suma de variables aleatorias gausianas (para cualquier hipótesis), también es gausiana:
\else Since $T$ is a sum of Gaussian random variables (for any hypothesis), it is Gaussian, too: 
\fi
\begin{align*}
p_{T|H}(t|0)\sim G\left (m_0, v_0 \right)   \\
p_{T|H}(t|1)\sim G\left (m_1, v_1 \right)
\end{align*}
\ifspanish donde \else where \fi
\begin{align*}
m_0 &= \EE\{T \mid H=0\} = \EE\{X_1 \mid H=0\} + \EE\{X_2 \mid H=0\} = 0   \\
m_1 &= \EE\{T \mid H=1\} = \EE\{X_1 \mid H=1\} + \EE\{X_2 \mid H=1\} = 2   \\
v_0 &= \EE\{(T-m_0)^2 \mid H=0\} = \EE\{X_1^2 + X_2^2 + 2 X_1 X_2 \mid H=0\} 
     = 1 + 1 + 0 = 2   \\
v_1 &= \EE\{(T-m_1)^2 \mid H=1\} = \EE\{(X_1-1) + (X_2-1))^2 \mid H=1\}  \\
    &= \EE\{(X_1-1)^2 + (X_2-1)^2 + 2 (X_1-1)(X_2-1) \mid H=1\}  = 1+1+0 = 1
\end{align*}
\ifspanish Por tanto, \else Therefore, \fi
\begin{align*}
p_{T|H}(t|0)\sim G\left ( 0, 2 \right)   \\
p_{T|H}(t|1)\sim G\left ( 2, 2 \right)
\end{align*}
\part 
\begin{align*}
\pmis &= \int_{-\infty}^1 p_{T|H}(t \mid 1) dt  
       = \int_{-\infty}^1 \frac1{\sqrt{4 \pi}} \exp\left(-\frac14 (t-2)^2 \right) dt  \\
      &= \int_{-\infty}^\frac1{\sqrt{2}} \frac1{\sqrt{2 \pi}} \exp\left(-\frac14 z^2 \right) dz
       = F\left (-\frac1{\sqrt{2}}\right)   \\
\pfa  &= \int_1^\infty p_{T|H}(t \mid 0) dt 
       = \int_1^\infty \frac1{\sqrt{4 \pi}} \exp\left(-\frac14 t^2 \right) dt   \\
      &= \int_{\frac1{\sqrt{2}}}^\infty \frac1{\sqrt{2 \pi}} \exp\left(-\frac12 t^2 \right) dt
       = 1 - F\left (\frac1{\sqrt{2}}\right)
\end{align*}
\ifspanish Nótese que, siendo $F(-z)=1-F(z)$, para todo $z \in \mathbb{R}$, resulta $\pfa=\pmis$.
\else Note that, since $F(-z)=1-F(z)$, for any $z \in \mathbb{R}$, we have $\pfa=\pmis$.
\fi
\end{parts}
\end{solution}
%%%%%%%%%%%%%%
