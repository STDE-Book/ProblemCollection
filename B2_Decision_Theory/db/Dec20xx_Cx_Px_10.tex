\ifspanish

\question Se tiene un problema de decisión binaria definido por las siguientes verosimilitudes:
	 $$ \begin{array}{l} 
					   p_{X_1,X_2|H}(x_1,x_2 | 0) = G \left( {\bf 0}, \left[ \begin{array}{cc}  
					   1 & \rho \\ \rho & 1					   
					    \end{array}  \right] \right)\\ \\
					  p_{X_1,X_2|H}(x_1,x_2 | 1) = G \left( {\bf m}, \left[ \begin{array}{cc}  
					   1 & \rho \\ \rho & 1					   
					    \end{array}  \right] \right)	  
  \end{array}$$
siendo ${\bf m}=\left[ m,m \right] ^T$, con $m>0$ y $|\rho |<1$.
\begin{parts}
\part Sabiendo que $P_H(0)=P_H(1)$, obténgase el decisor bayesiano de mínima probabilidad de error. Represéntese en el plano $X_1-X_2$ la frontera de decisión obtenida.
\part Sobre el clasificador obtenido en a), compruébese que $Z= X_1+X_2$ es un estadístico suficiente para la decisión. Obténganse las verosimilitudes de $H=0$ y $H=1$ sobre la variable aleatoria $Z$, $p_{Z|H}(z|0)$ y $p_{Z|H}(z|1)$.
\part Calcúlense las probabilidades de falsa alarma, de pérdida y de error del decisor anterior; exprésense estas probabilidades utilizando la función
$$F(x) = 1- Q(x) = \int_{-\infty}^x \frac{1}{\sqrt{2\pi}} \exp{\left( - \frac{t^2}{2}\right) } \; dt$$
\part Analícese cómo varía la probabilidad de error con el valor de $\rho$; para ello, considérense los casos $\rho=-1$, $\rho=0$ y $\rho=1$. Indíquese sobre el plano $X_1-X_2$, para cada valor de $\rho$, cómo se distribuyen las verosimilitudes, y represéntese la frontera de decisión.
\end{parts}
\begin{solution}
\begin{parts}
\part $ x_1+x_2 \dunodcero m$
\part $ t \dunodcero m$ \\ \\  $ p_{Z|H}(z | 0) = G \left( 0, 2(1+\rho) \right) \quad \quad  p_{Z|H}(z | 1) = G \left( 2m, 2(1+\rho) \right) $
\part $ P_{\rm FA}=P_{\rm M}=P_{\rm e}=1-F \left( \displaystyle \frac{m}{\sqrt{2(1+\rho)}}  \right)$
\part $\mbox{Si} \; \rho \rightarrow -1:\: P_{\rm e}=0 
\quad \quad \mbox{Si} \; \rho= 0:\: P_{\rm e}=1-F \left( \displaystyle \frac{m}{\sqrt{2}}  \right) 
\quad \quad \mbox{Si} \; \rho \rightarrow 1:\: P_{\rm e}=1-F \left(\displaystyle \frac{m}{2}  \right)$
\end{parts}
\end{solution}

\else

\question We have a binary decision problem with likelihoods:
	 $$ \begin{array}{l} 
					   p_{X_1,X_2|H}(x_1,x_2 | 0) = G \left( {\bf 0}, \left[ \begin{array}{cc}  
					   1 & \rho \\ \rho & 1					   
					    \end{array}  \right] \right)\\ \\
					  p_{X_1,X_2|H}(x_1,x_2 | 1) = G \left( {\bf m}, \left[ \begin{array}{cc}  
					   1 & \rho \\ \rho & 1					   
					    \end{array}  \right] \right)	  
  \end{array}$$
with ${\bf m} = [m,m]^T$, where $m>0$, and $|\rho |<1$.
\begin{parts}
\part Knowing that $P_H(0)=P_H(1)$, obtain the Bayes' decider incurring in a minimum probability of error.  Plot the obtained decision boundary on the plane $X_1-X_2$.
\part For the classifier obtained in a), verify that $Z= X_1+X_2$ is a sufficient statistic for the decision. Obtain the likelihoods of hypotheses $H=0$ and $H=1$ over random variable  $Z$, $p_{Z|H}(z|0)$ and $p_{Z|H}(z|1)$.
\part Calculate the false alarm, missing, and error probabilities of the previous decider, expressing them in terms of function
$$F(x) = 1- Q(x) = \int_{-\infty}^x \frac{1}{\sqrt{2\pi}} \exp{\left( - \frac{t^2}{2}\right) } \; dt$$
\part Analyze how the probability of error changes with $\rho$; in order to do so, consider cases $\rho=-1$, $\rho=0$ , and $\rho=1$. Indicate, for each of these values of $\rho$, how the likelihoods and decision boundary look like on the plane with coordinate axis $X_1-X_2$.
\end{parts}

\begin{solution}
\begin{parts}
\part $ x_1+x_2 \dunodcero m$
\part $ z \dunodcero m$ \\ \\  $ p_{Z|H}(z | 0) = G \left( 0, 2(1+\rho) \right) \quad \quad  p_{Z|H}(z | 1) = G \left( 2m, 2(1+\rho) \right) $
\part $ P_{\rm FA}=P_{\rm M}=P_{\rm e}=1-F \left( \displaystyle \frac{m}{\sqrt{2(1+\rho)}}  \right)$
\part $\mbox{If} \; \rho \rightarrow -1:\: P_{\rm e}=0 
\quad \quad \mbox{If} \; \rho= 0:\: P_{\rm e}=1-F \left( \displaystyle \frac{m}{\sqrt{2}}  \right) 
\quad \quad \mbox{If} \; \rho \rightarrow 1:\: P_{\rm e}=1-F \left(\displaystyle \frac{m}{2}  \right)$
\end{parts}
\end{solution}

\fi