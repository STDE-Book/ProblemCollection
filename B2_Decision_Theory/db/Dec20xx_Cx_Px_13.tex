\ifspanish

\question Considérese el problema de decisión binaria descrito por:
  $$ \begin{array}{l} 
					   p_{X_1,X_2|H}(x_1,x_2 | 0) = \left\lbrace   \begin{array}{ll} \alpha x_2 & \quad 0<x_1< \displaystyle \frac{1}{4} \quad 0<x_2<1 \\
					   0 &  \quad  \mbox{en el resto}\end{array} \right.	   \\ \\
					  p_{X_1,X_2|H}(x_1,x_2 | 1) = \left\lbrace   \begin{array}{ll} \beta x_1 & \quad 0<x_1<1 \quad 0<x_2<  \displaystyle \frac{1}{2} \\
					   0 &  \quad  \mbox{en el resto}\end{array} \right.	
  \end{array}$$ 
 \begin{parts}
\part Tras obtener los valores de las constantes $\alpha$ y $\beta$, represéntense las regiones de decisión correspondientes a un decisor LRT. Indíquese cómo varían las regiones de decisión en función del umbral del clasificador. ?`Existe algún valor de dicho umbral para el que el clasificador obtenido sea lineal?
\part Obténganse las densidades de probabilidad marginales de $x_1$ y $x_2$ bajo ambas hipótesis ($H=0$ y $H=1$). ?`Qué relación estadística existe entre $X_1$ y $X_2$?
\part Por sencillez, se decide utilizar un detector de umbral basado en una única observación, de $X_1$ o de $X_2$:
$$ \mbox{DEC1:} \quad x_1 \dunodcero \eta_1 \quad \quad \quad \mbox{DEC2:} \quad x_2 \dceroduno \eta_2$$
Calcúlense las probabilidades de falsa alarma y de detección de los clasificadores DEC1 y DEC2, expresándolas en función de los umbrales de dichos decisores: $\eta_1$ y $\eta_2$, respectivamente.
\part Dibújense las curvas características de operación (ROC) (i.e., las curvas que representan $P_{\rm D}$ en función de $P_{\rm FA}$) correspondientes a los decisores DEC1 y DEC2, y discútase cómo cambia el punto de operación de cada clasificador al modificar el valor del umbral correspondiente.
\part A la luz de los resultados obtenidos, ?`puede concluirse que alguno de los dos decisores propuestos, DEC1 o DEC2, sea superior al otro?.
\end{parts} 

\begin{solution}
\begin{parts}
\part $\alpha=8$ y $\beta=4$. \\
Donde $ p_{X_1,X_2|H}(x_1,x_2 | 0) $ o $ p_{X_1,X_2|H}(x_1,x_2 | 1) $ son nulas se decide la hipótesis contraria. En la región donde ambas hipótesis no son nulas, considerando el LRT dado por $\displaystyle \frac{ p_{X_1,X_2|H}(x_1,x_2 | 0) }{ p_{X_1,X_2|H}(x_1,x_2 | 1) } \dceroduno \eta$, el decisor es:
 $$ 2x_2 - \eta x_1 \dceroduno 0$$
Para $\eta=4$ la frontera es lineal.
\part Las observaciones son independientes entre sí bajo ambas hipótesis. \\
$ p_{X_1|H}(x_1 | 0) = 4,  \quad 0<x_1< \displaystyle\frac{1}{4}  \quad \quad \; \; \;  p_{X_2|H}(x_2 | 0) =2 x_2, \quad 0<x_2<1 $\\
					$  p_{X_1|H}(x_1 | 1) = 2 x_1,  \quad 0<x_1<1 \quad \quad   p_{X_2|H}(x_2 | 1) =2, \quad 0<x_2< \displaystyle \frac{1}{2} $ 

\part 
$ \mbox{DEC1:} \left\lbrace \begin{array}{l}
\;  P_{\rm FA}= \left\lbrace \begin{array}{ll} 1-4\eta_1, & \; 0<\eta_1 < 1/4 \\
0, &  \; 1/4<\eta_1 <1 \end{array} \right.  \\ \\
 \;   P_{\rm D}=1-\eta_1^2, \quad 0<\eta_1 < 1 \end{array} \right. $ 
$ \mbox{DEC2:} \left\lbrace  \begin{array}{ll} 
\;  P_{\rm FA}=\eta_2^2, \quad 0<\eta_2 <1 \\ \\
\;   P_{\rm D}=\left\lbrace \begin{array}{ll} 2\eta_2, & \; 0<\eta_2 < 1/2 \\ 
						1, & \; 1/2 <\eta_2 < 1 \end{array} \right. \end{array} \right. $
\part 
DEC1: $\eta_1=1$ estamos en el punto $P_{\rm FA}=0$ y $P_{\rm D}= 0$, y si $\eta_1=0$ estamos en el punto $P_{\rm FA}=1$ y $P_{\rm D}= 1$.\\
DEC2: $\eta_2=1$ estamos en el punto $P_{\rm FA}=1$ y $P_{\rm D}= 1$, y si $\eta_2=0$ estamos en el punto $P_{\rm FA}=0$ y $P_{\rm D}= 0$.
\part No puede afirmarse que ninguno de los dos sea siempre mejor que el otro.
 		 
\end{parts} 
\end{solution}

\else

\question Consider a binary decision problem characterized by:
  $$ \begin{array}{l} 
					   p_{X_1,X_2|H}(x_1,x_2 | 0) = \left\lbrace   \begin{array}{ll} \alpha x_2 & \quad 0<x_1< \displaystyle \frac{1}{4} \quad 0<x_2<1 \\
					   0 &  \quad  \mbox{otherwise}\end{array} \right.	   \\ \\
					  p_{X_1,X_2|H}(x_1,x_2 | 1) = \left\lbrace   \begin{array}{ll} \beta x_1 & \quad 0<x_1<1 \quad 0<x_2<  \displaystyle \frac{1}{2} \\
					   0 &  \quad  \mbox{otherwise}\end{array} \right.	
  \end{array}$$ 
 \begin{parts}
\part After finding the values of constants $\alpha$ and $\beta$, provide a graphic representation of the decision regions corresponding to an LRT classifier. Indicate how those regions change as a function of the classifier threshold. Can the threshold be set so that the resulting classifier is linear?
\part Obtain the marginal probability density functions of $x_1$ and $x_2$ conditioned on both hypotheses ($H=0$ and $H=1$). What is the existing statistical relationship between $X_1$ and $X_2$?
\part For simplicity, we opt to use a threshold classifier based in just one variable: $X_1$ or $X_2$:
$$ \mbox{DEC1:} \quad x_1 \dunodcero \eta_1 \quad \quad \quad \mbox{DEC2:} \quad x_2 \dceroduno \eta_2$$
Calculate the probabilities of false alarm and detection of classifiers DEC1 and DEC2, expressing them as functions of the thresholds of such classifiers, $\eta_1$ and $\eta_2$, respectively.
\part Plot the ROC curves (i.e., the curves that represent $P_{\rm D}$ as a function of $P_{\rm FA}$), corresponding to deciders DEC1 and DEC2.  Discuss how the operation points of both classifiers change when modifying the corresponding threholds.
\part In the light of the obtained results, can it be concluded that one of the two proposed classifiers, DEC1 or DEC2, always outperforms the other one?
\end{parts} 

\begin{solution}
\begin{parts}
\part $\alpha=8$ and $\beta=4$. \\
We decide the only plausible hypothesis where $ p_{X_1,X_2|H}(x_1,x_2 | 0) $ or $ p_{X_1,X_2|H}(x_1,x_2 | 1) $ are zero. In the region where both likelihoods overlap, considering the LRT given by $\displaystyle \frac{ p_{X_1,X_2|H}(x_1,x_2 | 0) }{ p_{X_1,X_2|H}(x_1,x_2 | 1) } \dceroduno \eta$, the decider is:
 $$ 2x_2 - \eta x_1 \dceroduno 0$$
For $\eta=4$ a linear border is obtained.
\part Observations $X_1$ and $X_2$ are independent under both hypotheses. \\
$ p_{X_1|H}(x_1 | 0) = 4,  \quad 0<x_1< \displaystyle\frac{1}{4}  \quad \quad \; \; \;  p_{X_2|H}(x_2 | 0) =2 x_2, \quad 0<x_2<1 $\\
					$  p_{X_1|H}(x_1 | 1) = 2 x_1,  \quad 0<x_1<1 \quad \quad   p_{X_2|H}(x_2 | 1) =2, \quad 0<x_2< \displaystyle \frac{1}{2} $ 

\part 
$ \mbox{DEC1:} \left\lbrace \begin{array}{l}
\;  P_{\rm FA}= \left\lbrace \begin{array}{ll} 1-4\eta_1, & \; 0<\eta_1 < 1/4 \\
0, &  \; 1/4<\eta_1 <1 \end{array} \right.  \\ \\
 \;   P_{\rm D}=1-\eta_1^2, \quad 0<\eta_1 < 1 \end{array} \right. $ 
$ \mbox{DEC2:} \left\lbrace  \begin{array}{ll} 
\;  P_{\rm FA}=\eta_2^2, \quad 0<\eta_2 <1 \\ \\
\;   P_{\rm D}=\left\lbrace \begin{array}{ll} 2\eta_2, & \; 0<\eta_2 < 1/2 \\ 
						1, & \; 1/2 <\eta_2 < 1 \end{array} \right. \end{array} \right. $
\part 
DEC1: When $\eta_1=1$ the operation point is $P_{\rm FA}=0$ and $P_{\rm D}= 0$; for $\eta_1=0$ the operation point is $P_{\rm FA}=1$ and $P_{\rm D}= 1$.\\
DEC2: When $\eta_2=1$ the operation point is $P_{\rm FA}=1$ and $P_{\rm D}= 1$; for $\eta_2=0$ the operation point is $P_{\rm FA}=0$ and $P_{\rm D}= 0$.
\part None of the classifiers can be stated to always outperform the other.
 		 
\end{parts} 
\end{solution}

\fi