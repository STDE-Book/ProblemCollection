\ifspanish

\question[25] % JCS

Se desea averiguar si cierto cultivo celular prospera en un medio líquido determinado. Para ello, se mide la temperatura $X$ del cultivo (en grados centígrados) tras un tiempo $t > 1$ (medido en minutos). Se sabe que, cuando el cultivo prospera, la temperatura está dada por
$$ X = 10 \cdot  t \exp(-t) + R$$
% $$p_{X|H}(x|1) = {\cal N}(X-t^3\exp(-t), 1)$$
siendo $R$ una variable aleatoria de ruido gaussiano de media 0 y varianza 4.
% siendo ${\cal N}(t-m,v)$ la distribución gaussiana de media $m$ y varianza $v$.

Sin embargo, cuando el cultivo  no prospera, la temperatura evoluciona según
$$ X = 10 \exp(-t) + R$$
% $$p_{T|H}(t|0) = {\cal N}(t-\exp(-s), 1)    \qquad 0 \ge x \ge 1$$
A priori, la probabilidad de que el cultivo prospere es $P_H(1) = 0.5$. Se mide la temperatura tras $t$ minutos, y se desea determinar si el cultivo celular ha prosperado o no.
\begin{parts}
\part Determine la decisión de mínima probabilidad de error.
\part Determine la probabilidad de error. Exprese el resultado utilizando la función de distribución normalizada 
$$F(x) = \int_{-\infty}^{x} \dfrac{1}{\sqrt{2\pi}} 
                            \exp\left(-\dfrac{z^2}{2}\right) dz
$$
\part Determine cuánto tiempo debe esperarse para medir la temperatura, de tal modo que se minimice la probabilidad de error.
\part Transcurrido el tiempo obtenido en el apartado anterior, se mide una temperatura de 10 grados centígrados. Determine una expresión para la probabilidad de que el cultivo haya prosperado.
\end{parts}

\begin{solution}
\begin{parts}
\part $$X \dunodcero 5 (t+1) \exp(-t)$$
\part $$P_e = F\left(\dfrac{5}{2} (1-t) \exp(-t)\right)$$
\part $$t = 2$$
\part $$TBD$$
\end{parts}
\end{solution}

\else
\question[25] % JCS

We wish to find if a certain cell culture grows in a particular liquid environment. In order to do that, we measure the temperature $X$ of the culture (in Celsius degrees) after an elapsed time of $t > 1$ minutes. It is known that, if the culture is growing, the temperature is given by 
$$ X = 10 \cdot t \exp(-t) + R$$
% $$p_{T|H}(t|1) = {\cal N}(t-s^3\exp(-s), \exp(-s))    \qquad 0 \ge x \ge 1$$
%${\cal N}(t-m,v)$ representing the Gaussian distribution with mean $m$ and variance $v$.
where $R$ is a noise random variable with zero mean and variance 4.

However, when the cell culture does not evolve, the temperature is given by
$$ X = 10 \exp(-t) + R$$
% $$p_{T|H}(t|0) = {\cal N}(t-\exp(-s), \exp(-s))    \qquad 0 \ge x \ge 1$$
A priori, the probability that the cell culture grows is $P_H(1) = 0.5$. The temperature is measured after $t$ minutes, and we wish to decide whether the culture cell has grown or not.
% {\color{red}Yo pondría simplemente ${\cal N}(m,v)$}
\begin{parts}
\part Find the decision with minimum probability of error.
\part Find the probability of error of the previous classifier. Express your result in terms of the following normalized distribution function
$$F(x) = \int_{-\infty}^{x} \dfrac{1}{\sqrt{2\pi}} 
                            \exp\left(-\dfrac{z^2}{2}\right) dz
$$
\part Determine how long should we wait before measuring the temperature in order to minimize the probability of error.
\part After the time obtained in the previous section, a temperature $x=10$ degrees is observed. Find and expression for the probability that the cell culture has evolved.
\end{parts}

\begin{solution}
\begin{parts}
\part $$X \dunodcero 5 (t+1) \exp(-t)$$
\part $$P_e = F\left(\dfrac{5}{2} (1-t) \exp(-t)\right)$$
\part $$t = 2$$
\part $$TBD$$
\end{parts}
\end{solution}

\fi