\ifspanish

\question Un sistema genera dos observaciones, $X_1$ y $X_2$, que, tanto bajo hipótesis $H=0$ como $H=1$, son independientes e idénticamente distribuidas, siendo
 $$\begin{array}{ll}
 p_{X_i|H}(x_i|1)=2x_i & \quad 0<x_i<1 \\
 p_{X_i|H}(x_i|0)=2(1-x_i) & \quad 0<x_i<1
 \end{array}$$
Suponga hipótesis equiprobables.
\begin{parts}
\part Determine el decisor MAP basado en $X_1$ y calcule su probabilidad de error.
\end{parts}
Sea DMAP1 el decisor del apartado a), suponga que si $|x_1-0.5|<a$ (siendo $0<a<0.5$), se observa $X_2$ y, con objeto de seguir aplicando decisión por umbral, se descartan $X_1$ y la decisión de DMAP1. En su lugar, se aplica un segundo decisor, basado en $X_2$ y también MAP, que llamaremos DMAP2. 
\begin{parts}
\setcounter{partno}{1}
\part Represente gráficamente sobre el plano $X_1-X_2$, para un valor de $a$ arbitrario, las regiones de decisión del esquema conjunto DMAP1-DMAP2.
\part Determine la probabilidad de error global del esquema conjunto DMAP1-DMAP2.
\part Determine la máxima reducción de la probabilidad de error global que puede conseguirse utilizando el esquema conjunto, respecto al decisor DMAP1.
\part Compare las prestaciones del decisor conjunto DMAP1-DMAP2 con las del decisor MAP que utiliza simultáneamente $X_1$ y $X_2$.
\end{parts}
\begin{solution}
\begin{parts}
\part $ x_1 \dunodcero \displaystyle \frac{1}{2} \quad \quad P_{\rm e}=\displaystyle \frac{1}{4}$
\part 
$\begin{array}{ll}
D=0: & \;  x_1<1/2-a	\quad \mbox{y} \quad 1/2-a< x_1<1/2+a,  \; x_2<1/2 \\
D=1: & \;  1/2-a< x_1<1/2+a,  \;  x_2>1/2 	\quad \mbox{y} \quad  x1>1/2+a	\end{array}$
\part $P_{\rm e}=a^2-0.5a+0.25$
\part La variación máxima de la probabilidad de error es $\displaystyle \frac{1}{16}$
\part DMAP($X_1$ y $X_2$): $P_{\rm e}=\displaystyle \frac{1}{6}$ \\
DMAP1- DMAP2: $P_{\rm e}$ varía de $\displaystyle \frac{1}{4}$ a $\displaystyle \frac{1}{16}$
\end{parts}
\end{solution}

\else

\question A system generates two observations $X_1$ and $X_2$ that, under both hypothesis $H=0$ and $H=1$, are independent and identically distributed:
 $$\begin{array}{ll}
 p_{X_i|H}(x_i|1)=2x_i & \quad 0<x_i<1 \\
 p_{X_i|H}(x_i|0)=2(1-x_i) & \quad 0<x_i<1
 \end{array}$$
Assume that the {\em a priori} probability is the same for both hypotheses.
\begin{parts}
\part Determine the MAP decider based on $X_1$, and calculate its probability of error.
\end{parts}
Let DMAP1 be the decider of section a), and assume that if $|x_1-0.5|<a$ (with $0<a<0.5$), $X_2$ is also observed. When this happens, and with the goal of still applying a threshold classifier, $X_1$ is discarded (as well as DMAP1 decision, and a second MAP classifier (DMAP2), based on the observation of $X_2$, is applied. 
\begin{parts}
\setcounter{partno}{1}
\part Plot on plane $X_1-X_2$, for a generic value $a$, the decision regions for the joint scheme DMAP1-DMAP2.
\part Find the probability of error of the joint scheme DMAP1-DMAP2.
\part Find the maximum reduction of the probability of error that can be achieved using the joint scheme, with respect to the probability of error of decider DMAP1.
\part Compare the performance of the joint decider DMAP1-DMAP2 with that of the optimum MAP decider based on the joint observation of $X_1$ and $X_2$.
\end{parts}

\begin{solution}
\begin{parts}
\part $ x_1 \dunodcero \displaystyle \frac{1}{2} \quad \quad P_{\rm e}=\displaystyle \frac{1}{4}$
\part 
$\begin{array}{ll}
D=0: & \;  x_1<1/2-a	\quad \mbox{and} \quad 1/2-a< x_1<1/2+a,  \; x_2<1/2 \\
D=1: & \;  1/2-a< x_1<1/2+a,  \;  x_2>1/2 	\quad \mbox{and} \quad  x_1>1/2+a	\end{array}$
\part $P_{\rm e}=a^2-0.5a+0.25$
\part The maximum reduction of the probability of error is $\displaystyle \frac{1}{16}$
\part DMAP($X_1$ and $X_2$): $P_{\rm e}=\displaystyle \frac{1}{6}$ \\
DMAP1- DMAP2: $P_{\rm e}$ changes from $\displaystyle \frac{1}{4}$ to $\displaystyle \frac{1}{16}$
\end{parts}
\end{solution}

\fi