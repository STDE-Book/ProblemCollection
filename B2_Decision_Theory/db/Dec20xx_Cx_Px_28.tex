\ifspanish

\question Considere un problema de decisión binaria con hipótesis equiprobables y observaciones caracterizadas por
\begin{equation}
\begin{array}{ll} H=0: & X = N_0 \\ H=1: & X = a + N_1 \end{array}
\nonumber
\end{equation}
siendo $a$ una constante conocida y $N_0$ y $N_1$ variables aleatorias gaussianas con distribuciones $N_0 \sim G(0,v_0)$ y $N_1 \sim G(0,v_1)$, respectivamente.
\begin{parts}
\part Para $a>0$, ilustre gráficamente las regiones de decisión que se obtendrían en los casos $v_0 > v_1$, $v_0 < v_1$ y $v_0 = v_1$.
\part Considere para el resto del ejercicio $a=0$, $v_0 = 1$ y $v_1 = 2$. Obtenga la regla de decisión que minimiza la probabilidad de error del decisor.
\part Obtenga las probabilidades de falsa alarma y de detección que se obtienen al utilizar el decisor anterior. Exprese el resultado haciendo uso de la función $$F(u) = \int_{-\infty}^{u} \frac{1}{\sqrt{2\pi}} \exp\left(-\frac{u^2}{2}\right) du$$
\part Sobre una representación aproximada de la ROC de los decisores tipo LRT
\begin{equation}
\frac{p_{X|H}(x|1)}{p_{X|H}(x|0)} \dunodcero \eta   \nonumber
\end{equation}
indique cómo se desplazaría el punto de trabajo del decisor:
\begin{itemize}
\item al incrementar el umbral $\eta$ del decisor.
\item si crece la probabilidad a priori de la hipótesis $H=1$.
\end{itemize}
\end{parts}

\begin{solution}
\begin{parts}
\part Si $v_0 = v_1$ se obtendría un decisor de único umbral, en caso contrario se obtienen decisores con dos umbrales.
\part $|x| \dunodcero \sqrt{2 \ln 2}=x_u$
\part $\pfa = 2 F(-x_u), \quad \quad \pdet = 2 F\left(\dfrac{-x_u}{\sqrt{2}}\right)$
\part Si $\eta$ crece disminuyen $\pfa$ y $\pdet$. Si $P_H(1)$ crece, manteniendo $\eta$ constante, el punto de trabajo no varía.
\end{parts}
\end{solution}

\else

\question Consider a binary decision problem with equally probable hypotheses and observations characterized by
\begin{equation}
\begin{array}{ll} H=0: & X = N_0 \\ H=1: & X = a + N_1 \end{array}
\nonumber
\end{equation}
where $a$ is a known constant and $N_0$ and $N_1$ are Gaussian random variables with distributions $N_0 \sim G(0,v_0)$ and $N_1 \sim G(0,v_1)$, respectively.
\begin{parts}
\part For $a>0$, provide plots to illustrate the decision regions that would be obtained when $v_0 > v_1$, $v_0 < v_1$, and $v_0 = v_1$.
\part Consider during the rest of the exercise that $a=0$, $v_0 = 1$, and $v_1 = 2$. Obtain the decision rule that minimizes the probability of error of the decider.
\part Calculate the incurred probabilities of false alarm and detection when using the previous decider. Express your results by means of function $$F(z) = \int_{-\infty}^{z} \frac{1}{\sqrt{2\pi}} \exp\left(-\frac{u^2}{2}\right) du$$
\part Using an approximate representation of the ROC of LRT deciders
\begin{equation}
\frac{p_{X|H}(x|1)}{p_{X|H}(x|0)} \dunodcero \eta   \nonumber
\end{equation}
indicate how would the decider operation point move when:
\begin{itemize}
\item threshold $\eta$ is increased.
\item the {\em a priori} probability of hypothesis $H=1$ grows.
\end{itemize}
\end{parts}

\begin{solution}
\begin{parts}
\part If $v_0 = v_1$ we would obtain a classifier based on a single threshold over $x$; otherwise, there would be two thresholds.
\part $|x| \dunodcero \sqrt{2 \ln 2}=x_u$
\part $\pfa = 2 F\left (-x_u\right), \quad  \quad \pdet = 2 F\left ( \dfrac{-x_u}{\sqrt{2}}\right)$
\part If $\eta$ increases, then $\pfa$ and $\pdet$ decrease. If $P_H(1)$ increases with $\eta$ constant, the operation point does not change.
\end{parts}
\end{solution}

\fi