
\question

%% Use this if you have both Spanish and English versions
\ifspanish
Considere un problema de detección binaria ($H \in \{0,1\}$) y observaciones $X \in \mathbb{R}$. Las verosimilitudes son
\else
Consider a binary detection problem ($H \in \{0,1\}$) and observations $X \in \mathbb{R}$. The likelihoods are
\fi
\begin{align*}
  p_{X|H}(x|0) &= \frac{1}{\sqrt{2 \pi}} \exp \left(- \frac{x^2}{2}\right), \\
  p_{X|H}(x|1) &= \begin{cases} \frac{1}{2 a}, & -a < x < a, \\ 0, & \text{otherwise}, \end{cases}
\end{align*}
\ifspanish 
y las hipótesis son igualmente probables. Derive:
\else
and the hypotheses are equally likely. Derive:
\fi

\begin{parts}
\part
\ifspanish
Las regiones de decisión del detector que minimiza la probabilidad de error para un valor arbitrario de $a$, con $a > 0$.
\else
The decision regions of the detector that minimizes the probability of error for an arbitrary value of $a$, with $a > 0$.
\fi

\begin{solution}
\ifspanish El detector que minimiza la probabilidad de error está dado por
\else The detector that minimizes the probability of error is given by
\fi
\begin{equation*}
\frac{p_{X|H}(x|1)}{p_{X|H}(x|0)} \dunodcero \frac{P_{H}(0)}{P_{H}(1)} = 1,
\end{equation*}
\ifspanish que corresponde al detector de máxima verosimilitud para hipótesis igualmente probables. Antes de continuar, como siempre, es conveniente representar gráficamente estas verosimilitudes. Sin embargo, debemos considerar dos casos distintos:
\begin{enumerate}
  \item[A)] El valor máximo de $p_{X|H}(x|0)$ es mayor que el de $p_{X|H}(x|1)$, es decir,
  \begin{equation*}
    \frac{1}{2 a} > \frac{1}{\sqrt{2 \pi}} \Rightarrow a < \sqrt{\frac{\pi}{2}} \approx 1.25.
  \end{equation*}
  \item[B)] El valor máximo de $p_{X|H}(x|0)$ es menor (o igual) que el de $p_{X|H}(x|1)$, es decir,
  \begin{equation*}
    \frac{1}{2 a} \leq \frac{1}{\sqrt{2 \pi}} \Rightarrow a \geq \sqrt{\frac{\pi}{2}} \approx 1.25.
  \end{equation*}
\end{enumerate}
Entonces, para el Caso A) las verosimilitudes se muestran en la siguiente figura. A partir de ella, es fácil ver que
\begin{equation*}
|x| \dceroduno a,
\end{equation*}
y las regiones de decisión son
\else which is the maximum likelihood detector for equally likely hypotheses. Before proceeding, as always, it is convenient to plot these likelihoods. However, we need to consider two different cases:
\begin{enumerate}
  \item[A)] The largest value of $p_{X|H}(x|0)$ is larger than that of $p_{X|H}(x|1)$, i.e.,
  \begin{equation*}
    \frac{1}{2 a} > \frac{1}{\sqrt{2 \pi}} \Rightarrow a < \sqrt{\frac{\pi}{2}} \approx 1.25.
  \end{equation*}
  \item[B)] The largest value of $p_{X|H}(x|0)$ is smaller (or equal) than that of $p_{X|H}(x|1)$, i.e.,
  \begin{equation*}
    \frac{1}{2 a} \leq \frac{1}{\sqrt{2 \pi}} \Rightarrow a \geq \sqrt{\frac{\pi}{2}} \approx 1.25.
  \end{equation*}
\end{enumerate}
Then, for Case A) the likelihoods are shown in the following figure. From this figure, it is easy to see that
\begin{equation*}
|x| \dceroduno a,
\end{equation*}
and the decision regions are
\fi
\begin{align*}
\mathcal{X}_0 &= \{x \in \mathbb{R}  \mid |x| \geq a \},  \\
\mathcal{X}_1 &= \{x \in \mathbb{R}  \mid -a < x  < a \}.
\end{align*}
\begin{center}
	        \pgfmathsetmacro{\mypi}{3.141592}
	        \pgfmathsetmacro{\a}{1}
		\begin{tikzpicture}
		\begin{axis}[%
		axis x line=middle,
		axis y line=middle,
		enlarge x limits=0.05,
		enlarge y limits=0.5,
		xmin=-3,
		xmax=3,
		xtick={-\a,\a},
		xticklabels={$-a$,$a$},
		ymin=0,
		ymax=0.5,
		ytick={1/sqrt(2*\mypi),1/(2*\a)},
		yticklabels={$1/\sqrt{2 \pi}$,$1/2a$},
		ytick pos=left,
		yticklabel style = {yshift=0.2cm},
		width=10cm,
		height=7.5cm,
		samples = 512,
		xlabel={$x$},
		ylabel={$p_{X|H}(x|h)$}]
		\addplot[blue,thick,domain = -3:3] {1/sqrt(2*\mypi)*e^(-x^2/2)};
		\addlegendentry{$h = 0$};
		\addplot[red,thick,domain = -\a:\a] {1/(2*\a)};
		\addlegendentry{$h = 1$};
		\addplot +[mark=none,red] coordinates {(\a, 0) (\a, 1/(2*\a)};
		\addplot +[mark=none,red] coordinates {(-\a, 0) (-\a, 1/(2*\a)};
		\end{axis}
		\end{tikzpicture}
	\end{center}
\ifspanish Para el Caso B), las verosimilitudes se muestran en la figura siguiente, donde podemos ver que
\else For Case B), the likelihoods are shown in the next figure, where we can see that
\fi
\begin{align*}
\mathcal{X}_0 &= \{x \in \mathbb{R} \mid -b < x < b \} \cup \{x  \mid |x| \geq a \},  \\
\mathcal{X}_1 &= \{x \in \mathbb{R}  \mid b \leq |x| < a \},
\end{align*}
\ifspanish donde $b$ se obtiene como la solución positiva de \else where $b$ is obtained as the positive solution to \fi
\begin{equation*}
p_{X|H}(b|0) = p_{X|H}(b|1), \Rightarrow b = \sqrt{2 \log \left(\sqrt{\frac{2}{\pi}} a\right)}.
\end{equation*}
\ifspanish Por tanto, \else Then, we have \fi
\begin{align*}
\mathcal{X}_0 &= \left\{x \in \mathbb{R}  \, \left| \phantom{\frac{2}{\pi}} \right. -\sqrt{2 \log \left(\sqrt{\frac{2}{\pi}} a\right)}  < x < \sqrt{2 \log \left(\sqrt{\frac{2}{\pi}} a\right)} \right\} \cup \{x  \mid |x| \geq a \},   \\
\mathcal{X}_1 &= \left \{x \in \mathbb{R} \,  \left| \phantom{\frac{2}{\pi}} \right. \sqrt{2 \log \left(\sqrt{\frac{2}{\pi}} a\right)} \leq |x| < a \right\}.
\end{align*}
        
\begin{center}
	        \pgfmathsetmacro{\mypi}{3.141592}
	        \pgfmathsetmacro{\a}{2}
	        \pgfmathsetmacro{\b}{sqrt(2*ln(sqrt(2/\mypi)*\a))}
		\begin{tikzpicture}
		\begin{axis}[%
		axis x line=middle,
		axis y line=middle,
		enlarge x limits=0.05,
		enlarge y limits=0.5,
		xmin=-3,
		xmax=3,
		xtick={-\a,-\b,\b,\a},
		xticklabels={$-a$,$-b$,$b$,$a$},
		ymin=0,
		ymax=0.4,
		ytick={1/sqrt(2*\mypi),1/(2*\a)},
		yticklabels={$1/\sqrt{2 \pi}$,$1/2a$},
		ytick pos=left,
		yticklabel style = {yshift=0.2cm},
		width=10cm,
		height=7.5cm,
		samples = 512,
		xlabel={$x$},
		ylabel={$p_{X|H}(x|h)$}]
		\addplot[blue,thick,domain = -3:3] {1/sqrt(2*\mypi)*e^(-x^2/2)};
		\addlegendentry{$h = 0$};
		\addplot[red,thick,domain = -\a:\a] {1/(2*\a)};
		\addlegendentry{$h = 1$};
		\addplot +[mark=none,red] coordinates {(\a, 0) (\a, 1/(2*\a)};
		\addplot +[mark=none,red] coordinates {(-\a, 0) (-\a, 1/(2*\a)};
		\end{axis}
		\end{tikzpicture}
	\end{center}
  
\end{solution}

\part 
\ifspanish La probabilidad de detección, $\pdet$, como función de $a$, con $a >0$. Dibuja un gráfico de $\pdet$ frente a $a$ para $a \in (0,50)$.
\else The probability of detection, $\pdet$, as a function of $a$, with $a >0$. Sketch a plot of $\pdet$ vs. $a$ for $a \in (0,50)$.
\fi

\begin{solution}
\ifspanish Comencemos nuevamente con el Caso A). En este caso, la probabilidad de detección es
\else Let us start again with Case A). In this case, the probability of detection is \fi
  \begin{equation*}
    \pdet = P(D = 1 | H = 1) = \int_{\mathcal{X}_1} p_{X | H}(x | 1) d x = 1,
  \end{equation*}
\ifspanish
independientemente del valor de $a$, con $a < \sqrt{\frac{\pi}{2}}$. Es decir, para el Caso A) estamos integrando toda la verosimilitud bajo $H = 1$. Cuando consideramos el Caso B), el análisis se vuelve un poco más complejo. Concretamente, para $a \geq \sqrt{\frac{\pi}{2}}$, tenemos
\else
regardless of the value of $a$, with $a < \sqrt{\frac{\pi}{2}}$. That is, for Case A) we are integrating the whole likelihood under $H = 1$. When we consider Case B), it becomes a bit more involved. Concretely, for $a \geq \sqrt{\frac{\pi}{2}}$, we have
\fi
\begin{align*}
\pdet &= P(D = 1 | H = 1) = \int_{\mathcal{X}_1} p_{X | H}(x | 1) d x 
       = \int_{-a}^{-b} \frac{1}{2a} dx + \int_{b}^{a} \frac{1}{2a} dx = 2 \frac{a - b}{2 a} \\ 
      &= 1 - \frac{b}{a} = 1 - \frac{1}{a} \sqrt{2 \log \left(\sqrt{\frac{2}{\pi}} a\right)}.
\end{align*}
\ifspanish 
El gráfico de $\pdet$ se muestra en la figura siguiente:
\else the plot of $\pdet$ is shown in the following figure
\fi
\begin{center}
	        \pgfmathsetmacro{\mypi}{3.141592}
		\begin{tikzpicture}
		\begin{axis}[%
		axis x line=middle,
		axis y line=middle,
		enlarge x limits=0.05,
		enlarge y limits=0.1,
		xmin=0,
		xmax=50,
		%xtick={-\a,-\b,\b,\a},
		%xticklabels={$-a$,$-b$,$b$,$a$},
		ymin=0,
		ymax=1,
		%ytick={1/sqrt(2*\mypi),1/(2*\a)},
		%yticklabels={$1/\sqrt{2 \pi}$,$1/2a$},
		%ytick pos=left,
		%yticklabel style = {yshift=0.2cm},
		width=10cm,
		height=7.5cm,
		samples = 512,
		xlabel={$a$},
		ylabel={$\pdet(a)$}]
		\addplot[blue,thick,domain = 0.01:sqrt(\mypi/2)] {1};
		\addplot[blue,thick,domain = sqrt(\mypi/2):100] {1 - (1/x)*sqrt(2*ln(sqrt(2/\mypi)*x))};
		\addplot +[mark=none,blue,thick] coordinates {(sqrt(\mypi/2)+.2, 0.62) (sqrt(\mypi/2), 1.002)};
		\end{axis}
		\end{tikzpicture}
	\end{center}
\end{solution}

\part 
\ifspanish La probabilidad de error para $a = 1$. \else The probability of error for $a = 1$.
\fi

\begin{solution}
\ifspanish
Dado que $a = 1 < \sqrt{\pi/2}$, estamos en el Caso A), para el cual ya sabemos que $\pdet = 1$. Entonces, dado que
\begin{equation*}
\perr = \pfa \cdot P_{H}(0) + \pmis \cdot P_{H}(1) = \frac{1}{2} \pfa + \frac{1}{2} \left(1 - \pdet\right) = \frac{1}{2} \pfa,
\end{equation*}
solo queda calcular $\pfa$. Para el Caso A), la probabilidad de falsa alarma está dada por
\begin{equation*}
\pfa = P(D = 1 | H = 0) = \int_{\mathcal{X}_1} p_{X | H}(x | 0) d x = \int_{-a}^{a} \frac{1}{\sqrt{2 \pi}} \exp \left(- \frac{x^2}{2}\right) d x.
\end{equation*}
Teniendo en cuenta ahora la simetría de la verosimilitud, $\pfa$ se simplifica a
\begin{equation*}
\pfa = 2 \int_{0}^{a} \frac{1}{\sqrt{2 \pi}} \exp \left(- \frac{x^2}{2}\right) d x = 2 \left[\frac{1}{2} - \int_{a}^{\infty} \frac{1}{\sqrt{2 \pi}} \exp \left(- \frac{x^2}{2}\right) d x\right] = 1 - 2 Q(a),
\end{equation*}
lo que da como resultado
\begin{equation*}
\perr = \frac{1}{2} - Q(a).
\end{equation*}

\else
Since $a = 1 < \sqrt{\pi/2}$, we are in Case A), for which we already know that $\pdet = 1$. Then, since
  \begin{equation*}
    \perr = \pfa \cdot P_{H}(0) + \pmis \cdot P_{H}(1) = \frac{1}{2} \pfa + \frac{1}{2} \left(1 - \pdet\right) = \frac{1}{2} \pfa,
  \end{equation*}
  it only remains to compute $\pfa$. For Case A), the probability of false alarm is given by
  \begin{equation*}
    \pfa = P(D = 1 | H = 0) = \int_{\mathcal{X}_1} p_{X | H}(x | 0) d x = \int_{-a}^{a} \frac{1}{\sqrt{2 \pi}} \exp \left(- \frac{x^2}{2}\right) d x.
  \end{equation*}
  Taking now into account the symmetry of the likelihood, $\pfa$ simplifies to
  \begin{equation*}
    \pfa = 2 \int_{0}^{a} \frac{1}{\sqrt{2 \pi}} \exp \left(- \frac{x^2}{2}\right) d x = 2 \left[\frac{1}{2} - \int_{a}^{\infty} \frac{1}{\sqrt{2 \pi}} \exp \left(- \frac{x^2}{2}\right) d x\right] = 1 - 2 Q(a),
  \end{equation*}
  which yields
  \begin{equation*}
    \perr = \frac{1}{2} - Q(a).
  \end{equation*}
\fi  
\end{solution}
  
\end{parts}
