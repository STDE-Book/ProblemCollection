
\ifspanish

%%%%%%%%%
\question 

Las variables aleatorias $X$, $Y$ y $Z$ son estadísticamente independientes y siguen una distribución uniforme:
\begin{align*}
p_X(x) &= 1,    \qquad 0 \le x \le 1  \\
p_Y(y) &= 1,    \qquad 0 \le y \le 1  \\
p_Z(z) &= 1,    \qquad 0 \le z \le 1
\end{align*}

Considere los 3 problemas de decisión siguientes. En todos ellos, se observa $X$, pero no se conocen $Y$ ni $Z$.

Problema 1: dado por las hipótesis:
\begin{align*}
H=1: & \qquad  X > 0.2   \\
H=0: & \qquad  X \le 0.2
\end{align*}
Problema 2: dado por las hipótesis:
\begin{align*}
H=1: & \qquad  X > Y   \\
H=0: & \qquad  X \le Y
\end{align*}
Problema 3: dado por las hipótesis:
\begin{align*}
H=1: & \qquad (X > Y)   \quad \text{ y }\quad (X > Z) \\
H=0: & \qquad (X \le Y) \quad \text{ o }\quad (X \le Z)
\end{align*}

\begin{parts}
\part Determine el decisor MAP para el problema 1.
\part Determine la probabilidad de error del decisor MAP.
\part Determine el decisor MAP para el problema 2.
\part Determine la probabilidad de error del decisor anterior.
\part Determine el decisor MAP para el problema 3.
\part (Difícil) Determine la probabilidad de error del decisor anterior.
\end{parts}

%%%%%%%%%%%%%%%%
\begin{solution}

\begin{parts}
\part Observar $X=x$ implica conocer la hipótesis correcta: si $x>0.2$, la hipótesis correcta es 1, y será 0 en caso contrario. Por tanto, el decisor MAP está dado por
\begin{align*}
x \dunodcero 0.2
\end{align*}
Formalmente, esto también puede comprobarse aplicando directamente la expresión del decisor MAP:
\begin{align*}
P_{H|X}(1|x) & \dunodcero \frac12  \\
  \Leftrightarrow & \quad P\{X > 0.2 | X=x\} \dunodcero \frac12 \\
  \Leftrightarrow & \quad 
    \left[\begin{array}{ll}
          1 & \text{ si } x > 0.2  \\
          0  & \text{ si } x \le 0.2 
          \end{array} \right]
  \dunodcero \frac12 \\
  \Leftrightarrow & \quad 
    x \dunodcero 0.2 \\
\end{align*}

\part Dado que observar $x$ implica conocer $H$ de forma determinista, la probabilidad de error será cero. Formalmente, esto puede determinarse sabiendo que:
\begin{align*}
\pfa = P\{D=1|H=0\} = P\{X > 0.2| X \le 0.2\} = 0   \\
\pmis  = P\{D=0|H=1\} = P\{X \le 0.2| X > 0.2\} = 0
\end{align*}
luego $P_e = P_H(0)\pfa + P_H(1)\pmis = 0$

\part
Sabiendo que 
\begin{align*}
P_{H|X}(1|x) 
   &= P\{X > Y | X=x\} = P\{Y < x \} 
    = \int_0^x 1 dx = x 
\end{align*}
el decisor MAP será
\begin{align*}
x \dunodcero \frac12
\end{align*}

\part La probabilidad de error será
\begin{align*}
P_e 
  &= P\{D=1, H=0\} + P\{D=0, H=1\}        \\
  &= P\left\{X > \frac12, X \le Y \right\}
   + P\left\{X \le \frac12, X > Y \right\}\\
  &= \int_0^1 P\left\{X>\frac12, X\le Y |X=x\right\}p_X(x)dx
   + \int_0^1 P\left\{X\le\frac12, X> Y |X=x\right\}p_X(x)dx  \\
  &= \int_0^1 P\left\{x>\frac12, Y\ge x |X=x\right\} dx
   + \int_0^1 P\left\{x\le\frac12, Y<x |X=x\right\} dx  \\
  &= \int_\frac12^1 P\left\{Y\ge x |X=x\right\} dx
   + \int_0^\frac12 P\left\{Y<x |X=x\right\} dx  \\
  &= \int_\frac12^1 (1-x) dx
   + \int_0^\frac12 x dx 
   = \frac14
\end{align*}

\part Sabiendo que
\begin{align*}
P_{H|X}(1|x) 
   &= P\{X > Y, X > Z | X=x\}
    = P\{Y < x, Z < x\}
    = P\{Y < x\}\cdot P\{Z < x\}
    = x^2
\end{align*}
el decisor MAP será
\begin{align*}
x \dunodcero \frac1{\sqrt{2}}
\end{align*}

\part La probabilidad de error será
\begin{align*}
P_e &= P\{D=1, H=0\} + P\{D=0, H=1\} 
\end{align*}
El segundo término se puede calcular de forma análoga al apartado (c):
\begin{align*}
P\{D=0, H=1\}
  &= P\left\{X < \frac1{\sqrt{2}}, X \ge Y, X \ge Z \right\}  \\
  &= \int_0^1 P\left\{X<\frac1{\sqrt{2}}, \,\, X\ge Y, \,\, X\ge Z
                      \mid X=x\right\}p_X(x)dx   \\
  &= \int_0^\frac1{\sqrt{2}} 
         P\left\{x<\frac1{\sqrt{2}}, \,\, Y\le x, \,\, Z\le x 
                 \mid X=x\right\} dx   \\
  &= \int_0^\frac1{\sqrt{2}} 
         P\left\{Y\le x \right\} 
         P\left\{Z\le x \right\} dx       \\
  &= \int_0^\frac1{\sqrt{2}} x^2 dx 
   = \left[\frac13 x^3 \right]_0^\frac1{\sqrt{2}} \\
  &= \frac1{6\sqrt{2}}
\end{align*}
El primer término puede obtenerse como sigue:
\begin{align*}
P\{D=1, H=0\} 
  &= P\{D=1\} - P\{D=1, H=1\}
\end{align*}
El primer término de esta nueva ecuación es
\begin{align*}
P\{D=1\} &= P\left\{X > \frac1{\sqrt{2}} \right\} 
   = \int_\frac1{\sqrt{2}}^1 p_X(x) dx 
   = 1 - \frac1{\sqrt{2}} 
\end{align*}
El segundo terminó será
\begin{align*}
P\{D=1, H=1\}
  &= P\left\{X > \frac1{\sqrt{2}}, X \ge Y, X \ge Z \right\}  \\
  &= \int_0^1 P\left\{X>\frac1{\sqrt{2}}, \,\, X\ge Y, \,\, X\ge Z
                      \mid X=x\right\}p_X(x)dx   \\
  &= \int_\frac1{\sqrt{2}}^1
         P\left\{Y\le x \right\} 
         P\left\{Z\le x \right\} dx       \\
  &= \int_\frac1{\sqrt{2}}^1 x^2 dx 
   = \left[\frac13 x^3 \right]_0^\frac1{\sqrt{2}} \\
  &= \frac13 - \frac1{6\sqrt{2}}
\end{align*}
Juntando los términos anteriores, resulta
\begin{align*}
P_e &= P\{D=1\} - P\{D=1, H=1\} + P\{D=0, H=1\}  \\
    &= 1 - \frac1{\sqrt{2}} 
     - \frac13 + \frac1{6\sqrt{2}} + \frac1{6\sqrt{2}}  \\
    &= \frac23\left(1 - \frac1{\sqrt{2}} \right)
\end{align*}

\end{parts}
\end{solution}
%%%%%%%%%%%%%%

\else 

%%%%%%%%%
\question 

The random variables $X$, $Y$ and $Z$ are statistically independent and follow a uniform distribution:
\begin{align*}
p_X(x) &= 1,    \qquad 0 \le x \le 1  \\
p_Y(y) &= 1,    \qquad 0 \le y \le 1  \\
p_Z(z) &= 1,    \qquad 0 \le z \le 1
\end{align*}

Consider the following decision problems. In all of them, $X$ is observed, but neither $Y$ nor $Z$ are known.

\textbf{Problem 1}: given by the hypotheses:
\begin{align*}
H=1: & \qquad  X > 0.2   \\
H=0: & \qquad  X \le 0.2
\end{align*}
\textbf{Problem 2}: given by the hypotheses:
\begin{align*}
H=1: & \qquad  X > Y   \\
H=0: & \qquad  X \le Y
\end{align*}
\textbf{Problem 3}: given by the hypotheses:
\begin{align*}
H=1: & \qquad (X > Y)   \quad \text{ and }\quad (X > Z) \\
H=0: & \qquad (X \le Y) \quad \text{ or }\quad (X \le Z)
\end{align*}

\begin{parts}
\part Determine the MAP decision-maker for problem 1.
\part Compute the error probability of the MAP decision-maker for problem 1.
\part Determine the MAP decision-maker for problem 2.
\part Compute the error probability of the MAP decision-maker for problem 2.
\part Determine the MAP decision-maker for problem 3.
\part (hard) Compute the error probability of the MAP decision-maker for problem 3.
\end{parts}


%%%%%%%%%%%%%%%%
\begin{solution}

\begin{parts}
\part Observing $X=x$ implies knowing the true hypothesis: if $x>0.2$, the true hypothesis is 1, and it will be 0 otherwise. Thus, the MAP decision-maker is given by
\begin{align*}
x \dunodcero 0.2
\end{align*}
Formally, this can also be verified by applying the MAP decision rule in a straightforward way:
\begin{align*}
P_{H|X}(1|x) & \dunodcero \frac12  \\
  \Leftrightarrow & \quad P\{X > 0.2 | X=x\} \dunodcero \frac12 \\
  \Leftrightarrow & \quad 
    \left[\begin{array}{ll}
          1 & \text{ si } x > 0.2  \\
          0  & \text{ si } x \le 0.2 
          \end{array} \right]
  \dunodcero \frac12 \\
  \Leftrightarrow & \quad 
    x \dunodcero 0.2 \\
\end{align*}

\part Since observing $x$ implies knowing $H$ deterministically, the error probability will be zero. Formally, this can be determined by knowing that:
\begin{align*}
\pfa = P\{D=1|H=0\} = P\{X > 0.2| X \le 0.2\} = 0   \\
\pmis = P\{D=0|H=1\} = P\{X \le 0.2| X > 0.2\} = 0
\end{align*}
thus $P_e = P_H(0)\pfa + P_H(1)\pmis = 0$

\part Knowing that 
\begin{align*}
P_{H|X}(1|x) 
   &= P\{X > Y | X=x\} = P\{Y < x \} 
    = \int_0^x 1 dx = x 
\end{align*}
the MAP decision maker will be
\begin{align*}
x \dunodcero \frac12
\end{align*}

\part The error probability is
\begin{align*}
P_e 
  &= P\{D=1, H=0\} + P\{D=0, H=1\}        \\
  &= P\left\{X > \frac12, X \le Y \right\}
   + P\left\{X \le \frac12, X > Y \right\}  \\
  &= \int_0^1 P\left\{X>\frac12, X\le Y |X=x\right\}p_X(x)dx
   + \int_0^1 P\left\{X\le\frac12, X> Y |X=x\right\}p_X(x)dx  \\
  &= \int_0^1 P\left\{x>\frac12, Y\ge x |X=x\right\} dx
   + \int_0^1 P\left\{x\le\frac12, Y<x |X=x\right\} dx  \\
  &= \int_\frac12^1 P\left\{Y\ge x |X=x\right\} dx
   + \int_0^\frac12 P\left\{Y<x |X=x\right\} dx  \\
  &= \int_\frac12^1 (1-x) dx
   + \int_0^\frac12 x dx 
   = \frac14
\end{align*}

\part Knowing that
\begin{align*}
P_{H|X}(1|x) 
   &= P\{X > Y, X > Z | X=x\}
    = P\{Y < x, Z < x\}
    = P\{Y < x\}\cdot P\{Z < x\}
    = x^2
\end{align*}
the MAP decision-maker is
\begin{align*}
x \dunodcero \frac1{\sqrt{2}}
\end{align*}

\part The error probability is
\begin{align*}
P_e &= P\{D=1, H=0\} + P\{D=0, H=1\} 
\end{align*}
The first term can be computed in a similar way to (c):
\begin{align*}
P\{D=0, H=1\}
  &= P\left\{X < \frac1{\sqrt{2}}, X \ge Y, X \ge Z \right\}  \\
  &= \int_0^1 P\left\{X<\frac1{\sqrt{2}}, \,\, X\ge Y, \,\, X\ge Z
                      \mid X=x\right\}p_X(x)dx   \\
  &= \int_0^\frac1{\sqrt{2}} 
         P\left\{x<\frac1{\sqrt{2}}, \,\, Y\le x, \,\, Z\le x 
                 \mid X=x\right\} dx   \\
  &= \int_0^\frac1{\sqrt{2}} 
         P\left\{Y\le x \right\} 
         P\left\{Z\le x \right\} dx       \\
  &= \int_0^\frac1{\sqrt{2}} x^2 dx 
   = \left[\frac13 x^3 \right]_0^\frac1{\sqrt{2}} \\
  &= \frac1{6\sqrt{2}}
\end{align*}
The second term can be computed as follows:
\begin{align*}
P\{D=1, H=0\} 
  &= P\{D=1\} - P\{D=1, H=1\}
\end{align*}
The first term in this new equation is
\begin{align*}
P\{D=1\} &= P\left\{X > \frac1{\sqrt{2}} \right\} 
   = \int_\frac1{\sqrt{2}}^1 p_X(x) dx 
   = 1 - \frac1{\sqrt{2}} 
\end{align*}
The second term is
\begin{align*}
P\{D=1, H=1\}
  &= P\left\{X > \frac1{\sqrt{2}}, X \ge Y, X \ge Z \right\}  \\
  &= \int_0^1 P\left\{X>\frac1{\sqrt{2}}, \,\, X\ge Y, \,\, X\ge Z
                      \mid X=x\right\}p_X(x)dx   \\
  &= \int_\frac1{\sqrt{2}}^1
         P\left\{Y\le x \right\} 
         P\left\{Z\le x \right\} dx       \\
  &= \int_\frac1{\sqrt{2}}^1 x^2 dx 
   = \left[\frac13 x^3 \right]_0^\frac1{\sqrt{2}} \\
  &= \frac13 - \frac1{6\sqrt{2}}
\end{align*}
Joining all the above term, we get
\begin{align*}
P_e &= P\{D=1\} - P\{D=1, H=1\} + P\{D=0, H=1\}  \\
    &= 1 - \frac1{\sqrt{2}} 
     - \frac13 + \frac1{6\sqrt{2}} + \frac1{6\sqrt{2}}  \\
    &= \frac23\left(1 - \frac1{\sqrt{2}} \right)
\end{align*}

\end{parts}
\end{solution}

\fi




