\ifspanish

\question Se lanza al aire un dado tradicional (caras con puntos de 1 a 6) y se genera la v.a. $X$ tal que
$$p_X(x) = \left\{\begin{array}{ll}
					\displaystyle
					\frac{2}{a} \left(1-\frac{x}{a}\right), & 0<x<a\\
					\\
					0, & {\mbox{en otro caso}}  
	 			  \end{array} 
           \right. $$
de modo tal que su media viene dada por el resultado del lanzamiento (es igual a los puntos que muestra la cara de arriba).

Sup\'{o}ngase que, para una tirada, se tiene acceso a 3 medidas del valor de $X$ tomadas independientemente, de valores $x^{(1)} = 2, x^{(2)} = 5, x^{(3)} = 10$. Dec\'{\i}dase a partir de ellas el resultado del lanzamiento del dado seg\'{u}n el criterio de m\'{a}xima verosimilitud.

\begin{solution}
El criterio de m\'{a}xima verosimilitud determina que se ha de elegir la cara 5.
\end{solution}

\else

\question A fair dice (with faces from 1 to 6) is thrown and the r.v. $X$ with pdf 
$$p_X(x) = \left\{\begin{array}{ll}
						  	\displaystyle
								  \frac{2}{a} \left(1-\frac{x}{a}\right), & 0<x<a\\
								  \\
								  0, & {\mbox{otherwise}}  
	 					 \end{array} 
						  \right.						  
				      $$
is generated so that its mean is given by the result of throwing the dice (i.e., the mean is equal to the number of points in the upper face).
Assume that for a given throw we have access to 3 independent measurements of $X$, with values $x^{(1)} = 2, x^{(2)} = 5, x^{(3)} = 10$.
Decide from these values which is the result of throwing the dice according to the maximum likelihood criterion. 


\begin{solution}
The Maximum Likelihood criterion determines that face `5' should be selected.
\end{solution}

\fi