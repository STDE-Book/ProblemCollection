\ifspanish

\question[25] % VGV

El encargado de una empresa informática desea analizar la productividad de sus empleados mediante la estimación del tiempo, $S$, que les lleva desarrollar cierto programa informático. Para ello, a las 12:00 a.m. el encargado solicita a sus empleados la realización de este programa. Los empleados no atienden la petición hasta que finalicen la tarea que estén llevando a cabo en ese momento, lo que les lleva un tiempo adicional $N$. En consecuencia, el tiempo total transcurrido desde la petición del encargado hasta que el empleado termina el programa es $X=S+N$.

Se sabe que el tiempo $N$ puede modelarse mediante la siguiente distribución exponencial:
$$p_{N}(n) = a \exp \left(-a n \right)  \quad n>0,$$ mientras que $S$ puede modelarse mediante una exponencial retardada, es decir:
$$p_{S}(s) = b \exp \left(-b  \left(s-c \right)    \right)  \quad s>c.$$


\begin{parts}
\part Antes de iniciar el análisis de productividad sobre los empleados, se ha simulado sobre un grupo de control, midiendo directamente sobre este grupo los tiempos que han tardado en acabar las tareas que están realizando y los tiempos que han tardado en desarrollar el programa informático. Como consecuencia se tienen los siguientes conjuntos de observaciones independientes: $6$, $10$, $12$ y $20$ minutos  para el tiempo $N$ y $6$, $12$, $18$ y $36$ minutos para el tiempo $S$. Estime por máxima verosimilitud los valores de las constantes $a$, $b$ y $c$.
\end{parts}
Considere de ahora en adelante que $a=10$ minutos, $b=10$ minutos y $c=5$ minutos.
\begin{parts}
\addtocounter{partno}{1} 
\part Si cuando comienza el análisis de productividad, el encargado recibe la notificación de finalización del programa de tres empleados diferentes a las $12$:$25$, a las $12$:$30$ y a las $12$:$40$ a.m., estime por máxima verosimilitud el tiempo que ha tardado cada uno de estos empleados en realizar el programa.
\part ¿Qué tiempo estimaríamos que ha tardado cada uno de estos empleados si utilizásemos un estimador de mínimo error cuadrático medio?
\end{parts}

\begin{solution}
\begin{parts}
\part $\hat{a}_{\rm ML}=\dfrac{K}{\sum_{k=1}^K n^{(k)}} = \dfrac{1}{12}$ minutos$^{-1}$; \\
$\hat{c}_{\rm ML} = \min_k \left\{ s^{(k)} \right\} = 6$ minutos;\\ $\hat{b}_{\rm ML}=\dfrac{K}{\sum_{k=1}^K \left( s^{(k)} -\hat{c}_{\rm ML} \right)} = \dfrac{1}{12}$ minutos$^{-1}$.
\part $\hat{s}_{\rm ML}=x$. $\hat{s}_{\rm ML}(x=25)=25$, $\hat{s}_{\rm ML}(x=30)=30$,$\hat{s}_{\rm ML}(x=40)=40$.
\part $\hat{s}_{\rm MMSE}=\frac{x+5}{2}$. $\hat{s}_{\rm MMSE}(x=25)=15$, $\hat{s}_{\rm MMSE}(x=30)=17.5$,$\hat{s}_{\rm MMSE}(x=40)=22.5$.
\end{parts}
\end{solution}

\else

\question The manager of an IT company intends to analyze the productivity of his employees by estimating the time $S$ they need to implement a certain computer program. With this goal, at 12:00 a.m. the manager requests the implementation of the program; instead of directly starting the coding task, the employees need to finish first whatever task they are currently carrying out, what requires an additional time $N$. As a consequence, the total elapsed time between the request of the program implementation and each employee's notification indicating the conclusion of the task is $X=S+N$.

It is known that the time $N$ that the employees need to finish the tasks and start the program implementation can be modeled as the following exponential distribution
$$p_{N}(n) = a \exp \left(-a n \right)  \quad n>0,$$
whereas the time for coding the program, $S$, follows also an exponential distribution, in this case characterized by the expression
$$p_{S}(s) = b \exp \left(-b  \left(s-c \right)    \right)  \quad s>c.$$

\begin{parts}
\part Before the described process, a simulation has been carried out using a control group, and measuring directly the times $N$ and $S$ required by the members of this group. As a result of the test, four independent observations were obtained for each variable. Concretely, the four observations for $N$ were $6$, $10$, $12$, and $20$ minutes, whereas the observations for $S$ were $6$, $12$, $18$, and $36$ minutes. Based on these observations, estimate using maximum likelihood the values of constants $a$, $b$, and $c$.
\end{parts}
Consider in the following $a=10$ minutes, $b=10$ minutes, and $c=5$ minutes.
\begin{parts}
\addtocounter{partno}{1} 
\part For the actual productivity test, the manager receives notifications from three different employees indicating that they have finished the implementation of the program at $12$:$25$, $12$:$30$, and $12$:$40$ a.m.  Estimate using maximum likelihood the time that each employee needed for the implementation of the program.
\part Repeat the estimation of the previous subsection if a minimum mean square error estimator were used.
\end{parts}

\begin{solution}
\begin{parts}
\part $\hat{a}_{\rm ML}=\frac{K}{\sum_{k=1}^K n^{(k)}}= \dfrac{1}{12}$ minutes$^{-1}$;\\
$\hat{c}_{\rm ML}= \min_k \left\{ s^{(k)} \right\} = 6$ minutes;\\ 
$\hat{b}_{\rm ML}= \frac{K}{\sum_{k=1}^K \left( s^{(k)} -\hat{c}_{\rm ML} \right)} = \dfrac{1}{12}$ minutes$^{-1}$.
\part $\hat{s}_{\rm ML}=x$. $\hat{s}_{\rm ML}(x=25)=25$, $\hat{s}_{\rm ML}(x=30)=30$,$\hat{s}_{\rm ML}(x=40)=40$.
\part $\hat{s}_{\rm MMSE}=\frac{x+5}{2}$. $\hat{s}_{\rm MMSE}(x=25)=15$, $\hat{s}_{\rm MMSE}(x=30)=17.5$,$\hat{s}_{\rm MMSE}(x=40)=22.5$.
\end{parts}
\end{solution}

\fi