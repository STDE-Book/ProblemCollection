\ifspanish

\question Una variable aleatoria $X$ sigue una distribución exponencial unilateral con parámetro $a>0$:
$$p_X(x)=\frac{1}{a}\exp\left(-\frac{x}{a} \right) \quad x>0$$
Como se sabe, la media y varianza de $X$ están dadas por $a$ y $a^2$, respectivamente. 
\begin{parts}
\part Determínese el estimador de máxima verosimilitud de $a$, $\hat{A}_\text{ML}$, basado en un conjunto de $K$ observaciones independientes de la variable aleatoria X, $\left\lbrace X_k \right\rbrace_{k=0}^{K-1} $ .
\part Se propone un nuevo estimador basado en el anterior y que obedece a la expresión:
	 $$\hat{A}=c \cdot \hat{A}_\text{ML},$$
donde $0\leq c \leq 1$  es una constante que permite un reescalado del estimador ML.  Obténganse el sesgo al cuadrado, la varianza y el error cuadrático medio (MSE) del nuevo estimador, y represéntense todos ellos en una misma figura en función del valor de $c$.
\part Determínese el valor de $c$ que minimiza el MSE, $c^*$, y discútase su evolución conforme el número de observaciones disponibles aumenta. Calcúlese el MSE del estimador asociado a $c^*$.
\part Obténgase el intervalo de valores de $c$ para los que el MSE de $\hat{A}$ es menor que el MSE del estimador ML, y explíquese cómo varía dicho intervalo cuando $K\longrightarrow \infty$.  Discútase el resultado obtenido.
 \end{parts}
 
\begin{solution}
A video resolution of this problem (in Spanish) can be found in
 
\url{http://decisionyestimacion.blogspot.com/2013/05/problema-6-estimacion.html}


\begin{parts}
\part $\hat{A}_\text{ML}=\displaystyle\frac{1}{K} \sum_{k=1}^{K} {X_k}$
\part $\hat{A}=\displaystyle\frac{c}{K} \sum_{k=1}^{K} {X_k}$ \\
      $\EE\left\lbrace \hat{A}-a \right\rbrace^2=\left(c-1 \right)^2 a^2$, \\
      $\text{var} \left\lbrace \hat{A} \right\rbrace = \displaystyle\frac{c^2 a^2}{K}$, \\
      $\EE\left\{ \left(\hat{A}-a\right)^2 \right\} = \left(c-1 \right)^2 a^2 + \dfrac{c^2 a^2}{K}$
\part $c^*=\dfrac{K}{K+1}$, $c^*\rightarrow 1 \; (K\rightarrow \infty)$, \\ $\EE\left\{ \left( \hat{A}-a\right) ^2 \right\}=\dfrac{a^2}{K+1} \; (c=c^*)$
\part El intervalo de valores es: $c \in \left[ \dfrac{K-1}{K+1}, 1 \right]$, que se estrecha según aumenta $K$.
  \end{parts}
\end{solution}

\else

\question A random variable $X$ follows a unilateral exponential distribution with parameter $a>0$:
	 $$p_X(x)=\frac{1}{a}\exp\left(-\frac{x}{a} \right) \quad x>0$$
As it is known, the mean and variance of $X$ are given by $a$ and $a^2$, respectively. 

\begin{parts}
\part Obtain the maximum likelihood estimator of $a$, $\hat{A}_\text{ML}$, based on a set of $K$ independent observations of random variable $X$, $\left\{ X_k \right\}_{k=0}^{K-1} $.
\part Consider now a new estimator based on the previous one, and characterized by expression:
	 $$\hat{A}=c \cdot \hat{A}_\text{ML},$$
where $0\leq c \leq 1$  is shrinkage constant that allows re-scaling the ML estimator.  Find the bias squared, the variance, and the Mean Square Error (MSE) o the new estimator, and represent them all together in the same plot as a function of $c$.
\part Find the value of $c$ which minimizes the MSE, $c^*$, and discuss its evolution as the number of available observations increases.  Calculate the MSE of the estimator associated to $c^*$.
\part Determine the range of values of $c$ for which the MSE of $\hat{A}$ is smaller than the MSE of the ML estimator, and explain how such range changes as $K\longrightarrow \infty$.  Discuss your result.
 \end{parts}
 
\begin{solution}
\begin{parts}
\part $\hat{A}_\text{ML}=\displaystyle\frac{1}{K} \sum_{k=1}^{K} {X_k}$
\part $\hat{A}=\displaystyle\frac{c}{K} \sum_{k=1}^{K} {X_k}$ \\
      $\EE\left\lbrace \hat{A}-a \right\rbrace^2=\left(c-1 \right)^2 a^2$, \\
      $\text{var} \left\lbrace \hat{A} \right\rbrace = \displaystyle\frac{c^2 a^2}{K}$, \\
      $\EE\left\{ \left(\hat{A}-a\right)^2 \right\} = \left(c-1 \right)^2 a^2 + \dfrac{c^2 a^2}{K}$
\part $c^*=\dfrac{K}{K+1}$, $c^*\rightarrow 1 \; (K\rightarrow \infty)$, \\ $\EE\left\{ \left( \hat{A}-a\right) ^2 \right\}=\dfrac{a^2}{K+1} \; (c=c^*)$
\part The range of values is: $c \in \left[ \dfrac{K-1}{K+1}, 1 \right]$, which narrows as $K$ increases.
  \end{parts}
  \end{solution}

\fi