\ifspanish

%%%%%%%%%
\question 

Las variables aleatorias $S$ y $X$ están relacionadas a través de la distribución a posteriori:
\begin{align*}
p_{S|X}(s|x) = \frac{x + 4s - s^2}{x + \frac53}, \qquad 0 \le s \le 1, \qquad x \ge 0
\end{align*}
\begin{parts}
\part Determine el estimador de mínimo error cuadrático medio, $\SMSE$.
\part Determine el estimador de máximo a posteriori, $\SMAP$.
\end{parts}

\begin{solution}
\begin{parts}
\part 
\begin{align*}
\hat{s}_\text{MMSE}
    &= \mathbb{E}\{S|X=x\}   
     = \int_0^1 s \frac{x + 4s - s^2}{x + \frac53} ds  \\
    &= \frac{6 x + 13}
            {12 x + 20}
\end{align*}
\part 
\begin{align*}
\hat{s}_\text{MAP}
    &= \argmax_s \{p_{S|X}(s|x)\}  
     = \argmax_{s \in [0, 1]} \left\{\frac{x + 4s - s^2}{x + \frac53}\right\} \\
    &= \argmax_{s \in [0, 1]} \left\{x + 4s - s^2 \right\} \\
\end{align*}
La parábola $x + 4s - s^2$ tiene derivada
\begin{align*}
-2s + 4
\end{align*}
que se anula en $s=2$, que queda fuera del dominio de $p_{S|X}(s|x)$. Por tanto, $\hat{s}_\text{MAP}$ debe coincidir con uno de los extremos del intervalo $[0, 1]$. Dado que la derivada es positiva en ese intervalo, $p_{S|X}(s|x)$ es creciente en el dominio, y por tanto
\begin{align*}
\hat{s}_\text{MAP} = 1
\end{align*}

\end{parts}
\end{solution}


\else

%%%%%%%%%
\question   % JCS

The random variables $S$ and $X$ are related through the posterior distribution:
\begin{align*}
p_{S|X}(s|x) = \frac{x + 4s - s^2}{x + \frac53}, \qquad 0 \le s \le 1, \qquad x \ge 0
\end{align*}
\begin{parts}
\part Determine the least mean square error estimator, $\SMSE$.
\part Determine the maximum posterior estimator, $\SMAP$.
\end{parts}


\begin{solution}
\begin{parts}
\part
\begin{align*}
\SMSE &= \EE\{S|X=x\}
       = \int_0^1 s \frac{x + 4s - s^2}{x + \frac53} ds 
       = \frac{6x + 13}
              {12 x + 20}
\end{align*}
\part
\begin{align*}
\SMAP &= \argmax_s \{p_{S|X}(s|x)\}
       = \argmax_{s \in [0, 1]} \left\{\frac{x + 4s - s^2}{x + \frac53}\right\} \\
      &= \argmax_{s \in [0, 1]} \left\{x + 4s - s^2 \right\}
\end{align*}
The parabola $x + 4s - s^2$ has a derivative
\begin{align*}
-2s + 4
\end{align*}
which vanishes at $s=2$, which falls outside the domain of $p_{S|X}(s|x)$. Therefore, $\sMAP$ must coincide with one of the endpoints of the interval $[0, 1]$. Since the derivative is positive on that interval, $p_{S|X}(s|x)$ is increasing in the domain, and hence
\begin{align*}
\sMAP = 1
\end{align*}

\end{parts}
\end{solution}

\fi

